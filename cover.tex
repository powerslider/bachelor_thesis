% NOTE:
% These templates make an effort to conform to the MIT Thesis specifications,
% however the specifications can change.  We recommend that you verify the
% layout of your title page with your thesis advisor and/or the MIT 
% Libraries before printing your final copy.

%-----------University header----------
    \begin{figure}  
        \begin{subfigure}[b]{0.3\textwidth}
            \includegraphics[width=3.5cm, left]{tu-sofia-logo.png}
        \end{subfigure}
        \begin{subfigure}[b]{0.36\textwidth}
            \centering
            \Large{Technical University Sofia}
        \end{subfigure}
        %
        % \hspace{7cm}
        %
        \begin{subfigure}[b]{0.3\textwidth}
            \includegraphics[width=4cm, right]{fdiba-logo.png}
        \end{subfigure}
    \end{figure}
    \begin{center}
        \textbf{\large{Department of German Engineering and Industrial Management}}
        
        \vspace*{4cm}
        
        \textbf{\Huge{Integration of semantic technologies in the processing of news}}
    
        \vspace{1.5cm}

        \textbf{\large{Tsvetan Dimitrov}}
        
        \vspace{1.5cm}
        
        \textbf{\large{Supervisor: Assoc. Prof. Adelina Aleksieva}}

        
        \vfill
        
        A thesis presented for the degree of\\
        Bachelor of Informatics
        
        \vspace{0.8cm}
        
        \includegraphics[width=0.4\textwidth]{university}
    
        Sofia, Bulgaria\\
        July 4, 2016
    \end{center}
% -----------University header----------

% % The abstractpage environment sets up everything on the page except
% % the text itself.  The title and other header material are put at the
% % top of the page, and the supervisors are listed at the bottom.  A
% % new page is begun both before and after.  Of course, an abstract may
% % be more than one page itself.  If you need more control over the
% % format of the page, you can use the abstract environment, which puts
% % the word "Abstract" at the beginning and single spaces its text.

% %% You can either \input (*not* \include) your abstract file, or you can put
% %% the text of the abstract directly between the \begin{abstractpage} and
% %% \end{abstractpage} commands.

% % First copy: start a new page, and save the page number.
% \cleardoublepage
% % Uncomment the next line if you do NOT want a page number on your
% % abstract and acknowledgments pages.
% % \pagestyle{empty}
% \setcounter{savepage}{\thepage}
% \begin{abstractpage}
% %% The text of your abstract and nothing else (other than comments) goes here.
%% It will be single-spaced and the rest of the text that is supposed to go on
%% the abstract page will be generated by the abstractpage environment.  This
%% file should be \input (not \include 'd) from cover.tex.

The purpose of this thesis is to develop analytics methods for and visualisations of the news data from a public Ontotext service called News On the Web (NOW) using semantic technologies, standards of the Semantic Web and Linked Open Data. More specific tasks are generating a "News Map of Today's news" and recommending linked news and objects based on analyzing "hidden champions". Hidden champions are concepts which are not directly mentioned in the news, but are linked in the graph with a few concepts which are. For this purpose a cognitive technique called "priming" is used in which the popularity of the mentioned concepts is being spread out across the knowledge graph, following the principle of spreading activation in neural networks. The News Map will be visualised in the form of a word cloud in various options: direct popularity of the concepts based on mentions count in the news and degree of importance of these news, relative popularity - where the popularity within the current day is normalized against the popularity within the last year, concepts that are "hidden champions" and a combination of "hidden champions" and relative popularity. Analogous maps will be created for a separate news article as a way to visualise the important subjects in it.
% \end{abstractpage}

% Additional copy: start a new page, and reset the page number.  This way,
% the second copy of the abstract is not counted as separate pages.
% Uncomment the next 6 lines if you need two copies of the abstract
% page.
\setcounter{page}{\thesavepage}
\begin{abstractpage}
\thispagestyle{empty}
%% The text of your abstract and nothing else (other than comments) goes here.
%% It will be single-spaced and the rest of the text that is supposed to go on
%% the abstract page will be generated by the abstractpage environment.  This
%% file should be \input (not \include 'd) from cover.tex.

The purpose of this thesis is to develop analytics methods for and visualisations of the news data from a public Ontotext service called News On the Web (NOW) using semantic technologies, standards of the Semantic Web and Linked Open Data. More specific tasks are generating a "News Map of Today's news" and recommending linked news and objects based on analyzing "hidden champions". Hidden champions are concepts which are not directly mentioned in the news, but are linked in the graph with a few concepts which are. For this purpose a cognitive technique called "priming" is used in which the popularity of the mentioned concepts is being spread out across the knowledge graph, following the principle of spreading activation in neural networks. The News Map will be visualised in the form of a word cloud in various options: direct popularity of the concepts based on mentions count in the news and degree of importance of these news, relative popularity - where the popularity within the current day is normalized against the popularity within the last year, concepts that are "hidden champions" and a combination of "hidden champions" and relative popularity. Analogous maps will be created for a separate news article as a way to visualise the important subjects in it.
\end{abstractpage}

\cleardoublepage

\section*{Acknowledgments}

This is the acknowledgements section.  You should replace this with your
own acknowledgements.

%%%%%%%%%%%%%%%%%%%%%%%%%%%%%%%%%%%%%%%%%%%%%%%%%%%%%%%%%%%%%%%%%%%%%%
% -*-latex-*-
